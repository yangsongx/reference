多年之前我写过一篇书评《〈Word排版艺术〉读后感——兼谈与\LaTeX 的比较》
\footnote{\myurl{http://blog.csdn.net/solstice/article/details/187233}},
其中写道{\kaishu “如果将来有时间,我把自己用\LaTeX 排书的经验总结一下,
让读者在阅读《Word排版艺术》的基础上,更容易地把知识应用到\LaTeX 排版中去。”}
我自己排版了 \mybooktitle,现在终于可以把账还上了。
本文假定读者已经读过\LaTeX 的入门文档
\footnote{\myurl{http://mirrors.ctan.org/info/lshort/chinese/lshort-zh-cn.pdf}}
\footnote{\myurl{http://www.tex.ac.uk/tex-archive/info/latex-notes-zh-cn/latex-notes-zh-cn.pdf}}
和书籍
\footnote{《\LaTeX 入门与提高(第2版)》,陈志杰等著,高等教育出版社。},
具备基本的使用技能,这不是一篇入门教程。

排版是一门大学问,我只是一名技术图书的作者,有一些初步的 \LaTeX 使用经验。
我不是专家,出版印刷的行话也不怎么会说。
本文的目的是让有志于用\LaTeX 来排版自己书的人少走一些弯路。
换句话说,这篇文章是讲“我是怎么做的”,不是讲“哪种做法最好”。
另外,遇到\LaTeX 使用方面的问题请先阅读FAQ
\footnote{\myurl{http://www.newsmth.net/bbscon.php?bid=460\&id=282515}},再上CTeX论坛
\footnote{\myurl{http://bbs.ctex.org/forum.php} } 或水木社区TeX版 \nolinebreak
\footnote{\myurl{http://www.newsmth.net/bbsdoc.php?board=TeX}} 发帖询问,
不要给我写信。(我最多能回答我那本书里某个版面是如何排出来的,无法解答你的具体问题。)

最新版下载地址:\myurl{http://code.google.com/p/chenshuo/downloads/detail?name=typeset.pdf}

\LaTeX 源文件:\myurl{http://github.com/chenshuo/typeset}

\subsubsection{更新记录}
\begindot
\item[] 2013-02-04 初版
\myenddot
